\input{preamble}

\begin{document}
\begin{titlepage}
\include{Portada}
\end{titlepage}

\section{Introducción}

En la práctica de la ciencia de datos, una tarea común es encontrar la decisión óptima para un conjunto de múltiples elecciones con restricciones. Por ejemplo, \textit{en portfolio managment} se busca encontrar la \textit{mejor} combinación de acciones para invertir sujeto al presupuesto de un fondo de inversiones. Para este ejemplo, se desea encontrar qué activos $i$ comprar y cuánto comprar $x_i$ de cada uno, de tal manera que se maximice un función objetivo $f(\underline{x})$ sujeta a un presupuesto $M$ y a un conjunto de restricciones $A$. Para este ejemplo, es posible resolverlo con varios métodos de optimización que se han visitado en clase: optimización por gradiente descendiente, métodos quasi-newton, etc. Sin embargo, que pasaría si las soluciones de las variables $x_i$ fueran enteras? De igual forma, ¿qué pasaría si tuvieramos más de una función objetivo $f_{i}(\underline{x})$? Este problema se conoce como \textit{Multiobjective Mix Integer Programming} (MOILP) y es un problema NP-Hard. El presente trabajo se enfoca en explicar qué es MOILP, cómo se resuelve y una aplicación real para un caso de negocio de una empresa de fitness.

\section{Multiobjective Mix Integer Programming}

(Aquí va la explicación de MOILP):
\begin{itemize}
    \item Definición Formal
    \item Distintos tipos de MOILP: pareto, soluciones con peso
    \item Solución Técnica
    \item Solvers disponibles (Gurobi, COIN, etc)
\end{itemize}

\section{Ejemplo: Fitpass}
Fitpass es una empresa Méxicana que ofrece acceso a más de 1,500 estudios de fitness en todo el país, de los cuales ofrece 12 actividades $A$. Estas actividades $A_{j} \in A$ varían desde barre hasta yoga. La empresa ofrece dos planes de membresía que permiten a los usuarios asistir a un número limitado de clases $K$ por estudio $i$ al mes, lo único que varía entre uno y otro plan es el acceso a estudios "pro". Cada estudio tiene un ranking $r_i$ que va de 1 a 5, donde 5 es el mejor ranking, y este es calificado por los usuarios.

Un problema cómun que enfrenta un usuario es decidir a cuántas clases ir $x_{i}$ por estudio al mes por estudio, dado que el usuario sabe a cuántas $M$ clases va a ir al mes, las actividades $A_{j} \in A$ que le gustan como también en las actividades que no le gustan y el \textit{sentimiento} de distancia $d_{i}$ que tiene que recorrer para llegar al estudio. Lo que busca optimizar el usuario es minimizar la distancia que recorre, maximizar el ranking de los estudios a los que asiste y diversificar las actividades a las que asiste.

\subsection{Formulación del problema}

Sea $x_{i}$ el número de clases a las que asiste el usuario en el estudio $i$, $y_{i}$ una variable binaria que indica si el usuario asiste a la actividad $A_{j}$, $d_{i}$ el sentimiento de distancia que recorre el usuario para llegar al estudio, $r_{i}$ el ranking del estudio y $p_{i}$ una variable binaria que indica si el estudio $i$ es un estudio "pro" o no. 

Por sencillez de notación, $\underline{x}$ es el vector de variables $x_{i}$, $\underline{y}$ es el vector de variables $y_{i}$, $\underline{d}$ es el vector de variables $d_{i}$ y $\underline{r}$ es el vector de variables $r_{i}$. De igual forma, se introduce el producto de Hadamard en vectores, $\underline{x} \circ \underline{y}$, que es el vector que resulta de multiplicar elemento a elemento los vectores $\underline{x}$ y $\underline{y}$. 

Mencionado lo anterior, el problema de optimización que busca resolver el usuario es el siguiente:

\begin{equation*}
  \begin{aligned}
  \min_{x,y} \quad (\omega_1) \underline{x}^T \underline{d} - 
  (\omega_2) \underline{x}^T\underline{p}\circ \underline{r} - (\omega_3) \mathbf{1}^{T} \underline{y} \quad
  \textrm{s.a.} \quad & 
  x_i \leq K, \\
    & \sum_{i = 1}^{n}{x_i} = M.   \\
    & y_i \leq \sum_{j \in A_i} x_j, \\
    & \underline{x} \in \mathbf{Z}^n , \\
    & \underline{y} \in {0, 1}^{n}.
  \end{aligned}
  \end{equation*}

\subsection{Detalles de la técnica}


\subsection{Solución}

\section{Conclusiones}



\bibliography{BMBibTeX}
\bibliographystyle{ieeetr}
\end{document}